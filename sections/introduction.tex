\section{Introduction}
\setlength{\parindent}{5ex}
Dans le cadre du projet de S2 à EPITA, nous sommes poussés à mettre en pratique les différentes connaissances acquises en Travaux Pratiques et en cours, dans la réalisation d'un projet en groupe.
Le projet de notre studio gameHUB est la création d'un jeu d'horreur: \emph{Nyctalopia}.


À l'arrivée de la première soutenance, nous avons préféré nous concentrer sur l'aspect technique du jeu, le ``backend'', toute la partie que le ou les joueurs ne verraient pas, ce qui nous a permis pour cette deuxième date clé de notre projet, d'avancer plus rapidement afin d'obtenir des résultats plus visibles surtout dans les parties graphiques et jouables.

Pour la seconde soutenance, nous avons eu moins de temps, par rapport à la première soutenance, donc nous avons décidé principalement de régler des bogues pouvant être très génants pour l'utilisateur, ainsi que l'implémentation de quelques fonctionnalités. 

Nous présentons, dans ce rapport de soutenance, tout le progrès réalisé dans notre projet lors de cette deuxième période de travail, comment les tâches ont été réparties, et les objectifs que l'on envisage d'atteindre pour la prochaine soutenance.

\vfill
\noindent\makebox[\linewidth]{\rule{.8\paperwidth}{.6pt}}\\[0.2cm]
EPITA Toulouse - Projet S2 - 2022 \hfill Nyctalopia - gameHUB
\noindent\makebox[\linewidth]{\rule{.8\paperwidth}{.6pt}}

\newpage